\documentclass{article}
\usepackage[utf8]{inputenc}
\usepackage[spanish]{babel}
\usepackage{listings}


\begin{document}

\begin{titlepage}
    \begin{center}
        \vspace*{1cm}
            
        \Huge
        \textbf{Parcial 1 - Calistenia}
            
        \vspace{0.5cm}
        \LARGE
        Informática II
            
        \vspace{1.5cm}
            
        \textbf{Juan Pablo Areiza Jiménez}
            
        \vfill
            
        \vspace{0.8cm}
            
        \Large
        Despartamento de Ingeniería Electrónica y Telecomunicaciones\\
        Universidad de Antioquia\\
        Medellín\\
        Marzo de 2021
            
    \end{center}
\end{titlepage}

\tableofcontents
\newpage
\section{Introducción}\label{intro}
En el presente trabajo, se plantea una posible solución al desafío propuesto en clase, el cual consistía en llevar unos objetos de un estado inicial a un estado final, el estado inicial se encuentra dado por dos tarjetas de iguales dimensiones las cuales se encuentran sobre una superficie y debajo de una hoja de papel, por otro lado, en el estado final, las tarjetas se encuentran sobre la hoja de papel, formando una pirámide entre ellas. Para pasar del estado inicial al estado final, el usuario solo debe utilizar una de sus manos para lograrlo, para esto, se le otorga una serie de instrucciones sin ningún tipo de información adicional. En el desarrollo del trabajo, se presentan las instrucciones recibidas por el usuario, así como los aprendizajes obtenidos con el desarrollo de esta actividad.

\section{Instrucciones} \label{contenido}
\begin{itemize}
    \item Con una sola mano, tomar la hoja de papel que se encuentra sobre las tarjetas, al tomarla, asegúrese de no generar ningún tipo de imperfección en la hoja.
    \item Deposite suavemente la hoja sobre la mesa, asegurándose de que no se encuentre ningún elemento entre la mesa y la hoja de papel.
    \item Tome por los laterales las tarjetas que quedaron expuestas al retirar la hoja de papel, al tomarlas, mantenga las caras de las tarjetas unidas entre si y alineadas una con respecto a la otra, en caso de que al tomar las tarjetas estas no se encuentren así, por favor, acomodarlas para cumplir con los parámetros mencionados, puede ayudarse con la superficie de la mesa, para alinear las tarjetas una con respecto a la otra.
    \item A continuación, dirija las tarjetas hacia la posición en la cual se encuentra la, es importante que no suelte las tarjetas y las mantenga en la posición que se estableció en el paso anterior.
    \item Manteniendo la posición y sin soltar las tarjetas, ubíquelas verticalmente sobre la hoja, de forma que la base de las tarjetas hagan contacto con la hoja.
    \item Ubique los dedos de la mano de la siguiente forma, sin soltar las tarjetas:
    \begin{enumerate}
        \item Pulgar: En un lateral de forma que este haga contacto con ambas tarjetas.
        \item Índice: En el lado corto de las tarjetas que se encuentra en dirección contraria a la hoja, el dedo debe hacer contacto con ambas tarjetas.
        \item Medio, anular y meñique: Se posicionan en el lateral libre, es decir, el lateral que se encuentra en dirección opuesta a la posición del dedo pulgar.
    \end{enumerate}
    \item Todos los dedos a excepción del dedo índice, sujetarán solo la tarjeta que se encuentra más cercana a la palma de la mano que sostiene las tarjetas, manteniendo la posición indicada en el paso anterior, el dedo índice se mantendrá en contacto con ambas tarjetas sin alterar la posición que este tenía en el paso anterior.
    \item Desplace con suavidad la mano que sostiene las tarjetas, durante el desplazamiento de la mano, no debe mover la base de la tarjeta que no se encuentra sujetada por el dedo pulgar, mientras que la base de la otra tarjeta sí se puede mover, formando una pirámide. El movimiento se debe realizar hasta encontrar una posición de equilibrio de forma que las tarjetas mantengan la posición piramidal.
    \item Suelte las tarjetas y retire su mano con delicadeza, procurando que las tarjetas se mantengan en pirámide sobre la hoja de papel.
\end{itemize}


\section{Conclusión} \label{conclusiones}
Para plantear la solución a esta actividad, se requirió reducir la ambigüedad lo mejor posible, de forma que el usuario pudiese entender adecuadamente las instrucciones y no dar espacio a interpretaciones. Con el desarrollo de esta actividad, se aprende a brindar las instrucciones adecuadas para llegar a una solución, este aprendizaje, se extrapola al ámbito de la programación pues durante la programación, si los programas generados no están formulados para abarcar cualquier posibilidad, es decir que no son dinámicos, se puede esperar que el programa colapse en algún punto, pues los computadores no tienen una capacidad interpretativa que ayude a evitar esto. Por tal motivo, es necesario brindar un análisis profundo a la situación, para brindar una solución que abarque cualquier posibilidad.  

\end{document}
